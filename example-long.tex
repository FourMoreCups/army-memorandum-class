\documentclass{armymemo}
\begin{document}

%the following are required fields
\AMdept{DEPARTMENT OF THE ARMY}
\AMunit{Organizational Name/Title}
\AMaddress{Standardized Street Address}
\AMcity{City State 12345-1234}
\AMsymbol{OFFICE SYMBOL}
% Specify the date - \today or like \AMdate{January 1, 1970}
\AMdate{17 May 2013}
%\AMdate{\today}
\AMfor{MEMORANDUM FOR U.S. Army Command and General Staff College (ATZL),
  100 Stimson Avenue, Ft Leavenworth, KS 66027-1352}
\AMsubject{Using and Preparing a Memorandum}
\AMsethead

%see the class file for how \Ni works
\Ni See paragraph 2-2 (of AR 25-50) for when to use a memorandum.

\Ni Single space the text and double space between paragraphs and
subparagraphs. Insert two blank spaces after ending punctuation (period and
question mark). % seriously? maybe on a Selectric...
Insert two blank spaces after a colon. When numbering subparagraphs, insert two
blank spaces after parentheses.

\Ni When a memorandum has more than one paragraph, number the paragraphs
consecutively. When paragraphs are subdivided, designate first subdivisions
using lowercase letters of the alpabet and indent $1/4$ inch as shown below.

\Nii When a paragraph is subdivided, it must have at least two subparagraphs.

\Nii If there is a subparagraph ``a,'' there must be a subparagraph ``b.''

\Niii Designate second subdivisions by numbers in parentheses; for example, (1),
(2), and (3) and indent $1/2$ inch as shown.

\Niii Do not subdivide beyond the third subdivision.

\Niiii Do not indent any further than the second subdivision.

\Niiii Use (a), (b), (c), and so forth at this level.

\Ni Moby Dick, a classic American Novel, the text of which is public domain,
follows in several paragraphs to make a longer example:

\Nii Call me Ishmael. Some years ago- never mind how long precisely- having little or
no money in my purse, and nothing particular to interest me on shore, I thought
I would sail about a little and see the watery part of the world. It is a way I
have of driving off the spleen and regulating the circulation. Whenever I find
myself growing grim about the mouth; whenever it is a damp, drizzly November in
my soul; whenever I find myself involuntarily pausing before coffin warehouses,
and bringing up the rear of every funeral I meet; and especially whenever my
hypos get such an upper hand of me, that it requires a strong moral principle to
prevent me from deliberately stepping into the street, and methodically knocking
people's hats off- then, I account it high time to get to sea as soon as I can.
This is my substitute for pistol and ball. With a philosophical flourish Cato
throws himself upon his sword; I quietly take to the ship. There is nothing
surprising in this. If they but knew it, almost all men in their degree, some
time or other, cherish very nearly the same feelings towards the ocean with me.

\Nii There now is your insular city of the Manhattoes, belted round by wharves as
Indian isles by coral reefs- commerce surrounds it with her surf. Right and
left, the streets take you waterward. Its extreme downtown is the battery, where
that noble mole is washed by waves, and cooled by breezes, which a few hours
previous were out of sight of land. Look at the crowds of water-gazers there.

\Nii Circumambulate the city of a dreamy Sabbath afternoon. Go from Corlears Hook to
Coenties Slip, and from thence, by Whitehall, northward. What do you see?-
Posted like silent sentinels all around the town, stand thousands upon thousands
of mortal men fixed in ocean reveries. Some leaning against the spiles; some
seated upon the pier-heads; some looking over the bulwarks of ships from China;
some high aloft in the rigging, as if striving to get a still better seaward
peep. But these are all landsmen; of week days pent up in lath and plaster- tied
to counters, nailed to benches, clinched to desks. How then is this? Are the
green fields gone? What do they here?

\Nii But look! here come more crowds, pacing straight for the water, and seemingly
bound for a dive. Strange! Nothing will content them but the extremest limit of
the land; loitering under the shady lee of yonder warehouses will not suffice.
No. They must get just as nigh the water as they possibly can without falling
And there they stand- miles of them- leagues. Inlanders all, they come from
lanes and alleys, streets avenues- north, east, south, and west. Yet here they
all unite. Tell me, does the magnetic virtue of the needles of the compasses of
all those ships attract them thither?

\Nii Once more. Say you are in the country; in some high land of lakes. Take almost
any path you please, and ten to one it carries you down in a dale, and leaves
you there by a pool in the stream. There is magic in it. Let the most
absent-minded of men be plunged in his deepest reveries- stand that man on his
legs, set his feet a-going, and he will infallibly lead you to water, if water
there be in all that region. Should you ever be athirst in the great American
desert, try this experiment, if your caravan happen to be supplied with a
metaphysical professor. Yes, as every one knows, meditation and water are wedded
for ever.

\Ni POC for this memo is John Doe, who can be reached at \texttt{john@dev.null}
or (555) 555-1234.

\AMsigblock{John Doe}{CPT, SC}{SCCC 05-11}{%
  1. The Art of War, Sun Tzu, ~400BC\\
  2. Meditations, Marcus Aurelius, 180AD\\
  3. On War, Clausewitz, 1832}
\AMcf{%
  CDR\\
  CSM}


\end{document}
