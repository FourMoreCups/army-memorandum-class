\documentclass{armymemo}
\usepackage{lipsum}
\begin{document}

%the following are required fields
\AMdept{DEPARTMENT OF THE ARMY}
\AMunit{Organizational Name/Title}
\AMaddress{Standardized Street Address}
\AMcity{City State 12345-1234}
\AMsymbol{OFFICE SYMBOL}
\AMfor{MEMORANDUM FOR U.S. Army Command and General Staff College (ATZL), 100 Stimson Avenue, Ft Leavenworth, KS 66027-1352}
\AMdate{17 May 2013}
\AMsubject{Using and Preparing a Memorandum}
\AMsethead

%see the class file for how \Ni works
\Ni See paragraph 2-2 (of AR 25-50) for when to use a memorandum.

\Ni Single space the text and double space between paragraphs and
subparagraphs. Insert two blank spaces after ending punctuation (period and
question mark). % seriously? maybe on a Selectric...
Insert two blank spaces after a colon. When numbering subparagraphs, insert two
blank spaces after parentheses.

\Ni When a memorandum has more than one paragraph, number the paragraphs
consecutively. When paragraphs are subdivided, designate first subdivisions
using lowercase letters of the alpabet and indent $1/4$ inch as shown below.

\Nii When a paragraph is subdivided, it must have at least two subparagraphs.

\Nii If there is a subparagraph ``a,'' there must be a subparagraph ``b.''

\Niii Designate second subdivisions by numbers in parentheses; for example, (1),
(2), and (3) and indent $1/2$ inch as shown.

\Niii Do not subdivide beyond the third subdivision.

\Niiii Do not indent any further than the second subdivision.

\Niiii Use (a), (b), (c), and so forth at this level.

\AMsigblock{John Doe}{CPT, SC}{SCCC 05-11}{Encl:\\1. Enclosure 1}

\end{document}
